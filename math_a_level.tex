
\documentclass[a4paper,12pt]{article}
\usepackage{amsmath, amssymb, amsthm}
\usepackage{geometry}
\geometry{margin=1in}
\usepackage{graphicx}

\begin{document}

\title{เอกสารประกอบการติว A-Level คณิตศาสตร์}
\author{พี่ภีม วิศวะ คอม จุฬา}
\date{}
\maketitle

\section{เซต (Set)}

\subsection{ความรู้เบื้องต้นเกี่ยวกับเซต}
เซต (Set) หมายถึงกลุ่มของสิ่งต่าง ๆ ที่แน่นอนและเป็นจริงเสมอ เช่น กลุ่มของจำนวนนับที่ไม่เกิน 4 ถือว่าเป็นเซต 
เพราะสามารถระบุได้แน่นอนว่ากลุ่มนั้นประกอบด้วย $\{1,2,3,4\}$.

\subsection{สัญลักษณ์แทนเซต}
โดยทั่วไปนิยมเขียนสัญลักษณ์แทนเซตด้วยตัวอักษรภาษาอังกฤษพิมพ์ใหญ่ เช่น $A, B, C, \dots$

ตัวอย่างของเซตที่ใช้บ่อย:
\begin{itemize}
    \item $\mathbb{R}$ คือ เซตของจำนวนจริง
    \item $\mathbb{N}$ คือ เซตของจำนวนนับ
    \item $\mathbb{Z}$ คือ เซตของจำนวนเต็ม
    \item $\mathbb{Q}$ คือ เซตของจำนวนตรรกยะ
\end{itemize}

\subsection{การเขียนแทนเซต}
เซตสามารถเขียนแทนได้สองแบบ:
\begin{enumerate}
    \item \textbf{การเขียนแบบแจกแจงสมาชิก} เช่น
          \begin{itemize}
              \item $A = \{1,2,3,4,5\}$
              \item $B = \{\text{วันจันทร์, วันอังคาร, ..., วันอาทิตย์}\}$
          \end{itemize}
    \item \textbf{การเขียนแบบบอกเงื่อนไขของสมาชิก} เช่น
          \begin{itemize}
              \item $A = \{x \mid x < 10, x \in \mathbb{N} \}$
          \end{itemize}
\end{enumerate}

\subsection{ชนิดของเซต}
\begin{itemize}
    \item \textbf{เซตจำกัด} คือ เซตที่สามารถบอกจำนวนสมาชิกได้แน่นอน เช่น เซตของจังหวัดในประเทศไทย
    \item \textbf{เซตอนันต์} คือ เซตที่มีจำนวนสมาชิกไม่จำกัด เช่น เซตของจำนวนเต็มบวก
\end{itemize}

\subsection{เซตที่เท่ากันและสับเซต}
\textbf{นิยาม:} เซต $A$ และ $B$ จะเท่ากันก็ต่อเมื่อทั้งสองเซตมีสมาชิกเหมือนกันทุกตัว \\

\textbf{สับเซต:} ถ้า $A$ เป็นสับเซตของ $B$ เขียนแทนด้วย $A \subseteq B$

\section{ตรรกศาสตร์ (Logic)}
\subsection{ประพจน์}
ประพจน์ (Proposition) คือ ประโยคที่เป็นจริง (T) หรือเป็นเท็จ (F) อย่างใดอย่างหนึ่งเท่านั้น

ตัวอย่าง:
\begin{itemize}
    \item $7 > 3$ (T)
    \item เดือนสิงหาคมมี 30 วัน (F)
\end{itemize}

\subsection{ตัวเชื่อมตรรกศาสตร์}
\begin{itemize}
    \item และ ($\land$)
    \item หรือ ($\lor$)
    \item ถ้า...แล้ว ($\rightarrow$)
    \item ก็ต่อเมื่อ ($\leftrightarrow$)
    \item นิเสธ ($\neg$)
\end{itemize}

\subsection{ตารางค่าความจริง}
\begin{center}
\begin{tabular}{|c|c|c|c|c|}
\hline
$P$ & $Q$ & $P \land Q$ & $P \lor Q$ & $P \rightarrow Q$ \\
\hline
T & T & T & T & T \\
T & F & F & T & F \\
F & T & F & T & T \\
F & F & F & F & T \\
\hline
\end{tabular}
\end{center}

\section{ระบบจำนวนจริง (Real Number System)}
\subsection{โครงสร้างของระบบจำนวนจริง}
\begin{itemize}
    \item จำนวนจริง ($\mathbb{R}$) 
    \item จำนวนตรรกยะ ($\mathbb{Q}$) เช่น $\frac{1}{2}, 0.75, -3$
    \item จำนวนอตรรกยะ เช่น $\pi, \sqrt{2}$
    \item จำนวนเต็ม ($\mathbb{Z}$)
    \item จำนวนเต็มบวก ($\mathbb{Z}^+$)
    \item จำนวนเต็มลบ ($\mathbb{Z}^-$)
    \item จำนวนนับ ($\mathbb{N}$)
\end{itemize}

\section{แบบฝึกหัด}
\begin{enumerate}
    \item กำหนดให้ $A = \{1,2,3,4,5\}$ และ $B = \{3,4,5,6,7\}$ จงหา $A \cap B$
    \item ตรวจสอบว่า $p \rightarrow q$ เป็นสัจนิรันดร์หรือไม่
    \item จงหาสับเซตของเซต $\{a,b,c\}$
\end{enumerate}

\end{document}
